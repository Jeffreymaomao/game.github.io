\documentclass[a4paper, reprint, showkeys, nofootinbib,twoside]{revtex4-1}
\usepackage[utf8]{inputenc}
\usepackage[T1]{fontenc}
\usepackage{xcolor}
\usepackage{graphicx}
\usepackage{tikz}
\usepackage{import}
\usepackage{caption}
\usepackage{subcaption}
\usepackage{lipsum}
\usepackage{mathtools}
\usepackage{physics}
\usepackage{amsfonts}
\usepackage{amssymb}
\usepackage{amsmath}
\usepackage{amsthm}
\usepackage{hyperref}

\hypersetup{
	pdftex,
	colorlinks=true,
	linkcolor=blue,
	filecolor=magenta,
	urlcolor=blue,
	pdftitle={Article},
	pdfauthor={Author},
}
%\usepackage{fontspec}
%\usepackage{xeCJK}
%\setCJKfamilyfont{kai}{標楷體}
%==================================================================================
\begin{document}
	\title{Computer Graphics, Ray-Tracing}
	\author{Chang-Mao Yang}
	\email[Correspondence email address: ]{jeffrey0613mao@gmail.com}
	\affiliation{National Chung Cheng University, Department of Physics}
	\date{\today}
	
	\begin{abstract}
		\lipsum[1]
	\end{abstract}
	
	\keywords{first keyword, second keyword, third keyword}
	
	\maketitle
	%------------------------------------------------
	%--------------
	\section{WebGL} \label{sec:webgl}
	\subsection{Canvas Surface}
		We consider a subset $\mathbb{I} \subset \mathbb{R}^2$ defined as:
		 \begin{equation}
		 \mathbb{I} = \{ (x,y)\in \mathbb{Z}^2\,|\, 0\leq x< m, 0\leq y< n\},
		 \end{equation}
		 which represents the position of an image, where this image surface has a specified width $m$ and height $n$. Also, consider a color vector space $\mathbb{C}\subset \mathbb{R}^3$ defined as:
		 \begin{equation}
		 \mathbb{C} = \{ (r,g,b)\in \mathbb{R}^3\,|\, r,g,b \in [0,1]\}.
		 \end{equation}
		 To describe the color at any given point $(x, y)$ on the image, we define a vector function $\vec{f} : \mathbb{I}\to \mathbb{C}$. This function maps each point \((x, y)\) to a color vector, represented as:
		\begin{equation}
		\vec{f}(x, y) = \begin{pmatrix} r(x, y) \\ g(x, y) \\ b(x, y)\end{pmatrix},
		\end{equation}
		where $r(x, y)$, $g(x, y)$ and $b(x, y)$ denote the red, green, blue components of the color at point $(x, y)$, respectively. 
		
		For example we can graph the gradient color in two direction as following
		\begin{figure}[h]\centering
			\begin{subfigure}{0.4\linewidth}\centering
				\includegraphics[width=\textwidth]{img/gradient-x.png}
				\caption{Define $r(x,y) = g(x,y)=b(x,y) = x/m$.}\label{fig:sub1}
			\end{subfigure}
			\hspace{0.05\linewidth}
			\begin{subfigure}{0.4\linewidth}\centering
				\includegraphics[width=\textwidth]{img/gradient-y.png}
				\caption{Define $r(x,y) = g(x,y)=b(x,y) = y/n$.}\label{fig:sub2}
			\end{subfigure}
			\caption{Two different definition of $\vec{f}$, where the color only change with the direction of $x$ or $y$.}\label{fig:test}
		\end{figure}
	
	%--------------
	\section{Screen} \label{sec:screen}
	Now, we consider the screen position $\vec{p}_{screen}$ to be
	\begin{equation}
	\vec{P}_{\rm screen} = \left(x,y\right),\quad x\in \left[-\frac{w}{h}, \frac{w}{h}\right],\quad y\in [-1,1],
	\end{equation}
	see as image below
	
	\begin{figure}[h]
	\centering
		\import{./}{img/screenPosition.tex}
	\caption{Screen Coordinates (SC) space.}
	\end{figure}
	
	\begin{itemize}
	\item  Normalized Device Coordinates (NDC) spce
		\begin{verbatim}
// gl_FragCoord.x : [0, 1] 
// gl_FragCoord.y : [0, 1] 
// screenPos.x : [-1.0 ~ 1.0] * width / height
// screenPos.y : [-1.0 ~ 1.0] * 1.0
		\end{verbatim}
	\item Screen Coordinates (SC) space
		\begin{verbatim}
// screenPos.x : [-1.0 ~ 1.0] * width / height
// screenPos.y : [-1.0 ~ 1.0] * 1.0
		\end{verbatim}
	\item World Coordinates (WC) space
	\end{itemize}
	


		


	%------------------------------------------------
	%\begin{thebibliography}{4}
		%\bibitem{Griffiths} D. J. Griffiths, \textit{Introduction to Electrodynamics} (Cambridge University Press, Cambridge, 2017).
		%\bibitem{Fleming}  A. Bobrinha, Revista Brasileira de Lorem Ipsum \textbf{23}, 179 (2002).
		%\bibitem{Feynman} R. P. Feynman, R. B. Leighton and M. Sands,
	%\end{thebibliography}
	%------------------------------------------------
	
\end{document}%文章結束

